\section{Brief Research Plan}
\label{sec:timetable}
The mile stones and matching time consumption is planed as:
\begin{itemize}
\item
Modelling the multiple tasks problem for data centres considering energy consumption and physics recourse management. (6 months or less)
\item
To develop an adaptive accurate perdition of work load in data centres based on machine learning algorithm in Linux based data centre systems. The work needs to be conducted with sufficient data and the prediction method is expected to be tested in simulation and real circumstances (The cooperation between companies or data centres is needed). (6 months or more)
\item
Study around the machine learning algorithms \cite{ma}, including reinforcement learning and supervised learning, to realise the way to adaptive optimal tasks scheduling approach. Develop the algorithms and exam the performance in simulation based on real time data in Linux. (6 months)
\item
Investigate around different scheduling approaches, including game theory or queuing theory. Evaluation is needed for tasks scheduling in data centres real applications. Apply the proposed scheduling approach on real systems. There is always a ”gap” between theoretical design and A cooperation between data centres or product companies is highly expected. (6 months or more)
\item
Study around the trade-off between ease of programming and exibilities of the abstractions. In Jikes RVM environment, I will develop an approach for flexible language abstractions using compiler frameworks, like OpenCL and EPL. (6 month or less)
\item
Develop a dynamic energy-aware recompiler based on the proposed smart scheduler. Then whole energy-efficient approach (all three Themes) for data centres will be assembled. The performance of the proposed tool-chain will be tested and evaluated. (6 months) 
\item
Conduct all finished work, push forward for possible improvements. Prepare and write up the thesis. (6 months)
\end{itemize}
