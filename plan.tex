\section{Brief Research Plan}\label{sec:timetable}
The mile stones and matching time consumption is planed as:
\begin{itemize}
\item
Modelling the multiple tasks problem for data centres considering energy consumption and physics recourse management. (6 months)
\item
To develop an adaptive accurate perdition of work load in data centres based on machine learning algorithm. The work needs to be conducted with sufficient data and the prediction method is expected to be tested in simulation and real circumstances (The cooperation between companies or data centres is needed). (6 months)
\item
Study around the machine learning algorithms \cite{ma}, including reinforcement learning and supervised learning, to realise the way to adaptive optimal tasks scheduling approach. (6 months)
\item
Develop the algorithms and exam the performance in simulation based on real time data. (6 months or less)
\item
Investigate around different scheduling approaches, including game theory or queuing theory. Evaluation is needed for tasks scheduling in data centres real applications.(6 months or more)
\item
Apply the proposed scheduling approach on real systems. There is always a ”gap” between theoretical design and A cooperation between data centres or product companies is highly expected. (6 months)
\item
Conduct all finished work, push forward for possible improvements. Prepare and write up the thesis. (6 months)
\end{itemize}