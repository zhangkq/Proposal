\begin{abstract}
The landscape of computing is changing. Much of today's computation cycles are provided by data
centres that suffer from embarrassing energy inefficiencies:
only a small fraction of a typical data centre's power is spent on useful work.
These inefficiencies lead to global energy waste is measured in billions of pounds and tens of millions of metric tons of CO2. This has to change.

This proposal {\bf aims to improve the raw performance and energy efficiency of data centres through innovative compiler and runtime optimisation}. The underlying issue of data centres is that the computing systems are conservatively provisioned for rare utilisation peaks, leading to energy waste from under-utilised systems and the over-provisioning of hardware. This project aims to develop tools and methodologies that allow system software to precisely schedule hardware resources to reduce energy waste and dynamically tailor application tasks to achieve the best possible performance accordingly.

As a departure from prior work, this research investigates the use of predictive modelling techniques to predict live server traffic, enabling us to build new scheduling policies for precise resource scheduling. It will also create a novel, energy-aware compilation system to dynamically tailor applications on the fly, through recompilation, and replacing algorithms or data structures as the program runs. One of the key innovations I wish to explore is {\bf putting portability and adaptation at the core of this approach} by using machine learning to automate design space exploration and allowing self-adaptation of the system software over time.

As knowledge of the underlying platform, workloads and applications grows, \textbf{the system software will enhance performance and eliminate the energy waste of data centres}. The operating cost will be cut down, programs will run faster, and data centres will draw less power, reducing their negative environmental impact. These are ambitious goals which require that a considerable number of difficulties are addressed and overcome. As a result, there is enormous scope for innovation and novel research in this area, which will open up new possibilities and challenges as never before.

\end{abstract}
