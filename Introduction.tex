\section{Introduction}\label{ch:overview}
%Need of Energy-efficiency:
Data centres are the facilities for holding computers, communication equipments and huge data storage. As the infrastructure of the modern electronic world, there is a \$50 billion market with an exponential growth rate\cite{bigdatacentre}. However, the energy consumption has become the fundamental issue for data centres\cite{EPAreport}\cite{Energygov}. For instance, the UK spends 3\% of total domestic electricity usage on data centres in, which equals to one nuclear power plant\cite{globalactionplan}. The carbon emission of the world’s data centres is approximate to the carbon footprint of the entire Czech Republic and will be tripled by 2020\cite{GeSI}. With the increasing demand of data centres, the energy efficiency problems draw improving attractions among the world.
%Need for optimisation software
\\
\linebreak
Hardware engineers are striving to resolve the energy issue through power management techniques in processors\cite{chanandopp}. They developed various power management features, like dynamic voltage, frequency scaling and heterogeneous processors, which have been applied in system components. Unfortunately, there are still no clear clues for software to realise the hardware potential\cite{lookbackandfor}\cite{towardsenergyeff}. Therefore, the data centres are often spend less than 20\% of the entire power consumption to perform the useful work, i.e. almost 80\% of the data centres power usage is wasted\cite{thecaseforepc}. 
However, energy efficiency improvement for data centres is challenging. In detail, the hardware resources are often conservatively provisioned for rare utilisation peaks because of inability to predict the load peaks accurately\cite{energy-aware}\cite{towardsenergyeff}. Since machines operate on the peak power mode when it is unnecessary, this leads to energy waste in underutilised systems. On the other hand, server applications are currently hard-coded and optimised for a fully utilised system, but energy efficiency generally degrades outside that range\cite{towardsenergyeff}\cite{autodatacentre}. Thus, techniques that can tailor programs according to workload and resource changes are essential for energy efficiency\cite{towardsenergyeff}. Additionally, servers are truly multi-tasking environment, where thousands of tasks running at the same time, competing for shared resources, often with the support of virtualisation technology\cite{energy-effcloud}. This certainly creates huge difficulties for software power optimisation. The situation becomes more difficult as more and more heterogeneous hardware are used nowadays\cite{aview}. Towards energy efficiency data centres, a technique that evolves and adapts to the changing workload and resource and delivers scalable energy-efficient performance is needed.
\\
\linebreak
%Advantages of using machine learning algorithms
Machine learning provides the possibilities of implementing both of predictive and adaptive managing functions\cite{ma}. Machine learning has been proved for automatically static optimization of paralleling computation by Wang and O'Boyle\cite{wangf}\cite{wangs}. For a static analysis point of view, this intelligent program parallelism method achieves portability among different platforms without professional knowledge\cite{wangs}. It is possible to extend machine learning concepts on workload prediction and runtime scheduling, while adapting to changing environment, e.g. work load, hardware setup, and platforms. With efficient tasks scheduling, the performance of data centres can be improved with less energy usage by reducing recourse over provision.
