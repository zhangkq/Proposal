\section{Research Impacts}\label{sec:conclusion}
The improvement of the energy efficiency of data centres is a critical issue, which is highlighted by a recent UK government report\cite{ukgov}. The solution to this problem will certainly be a great benefit to companies like AMR\cite{ARM}, IBM\cite{IBM}, and Intel\cite{Intel}. Without solving such energy issue, the data centres have to be constrained by huge power requirement in the future. The proposed adaptive optimal tasks scheduling approach shows the possibilities to resolve the challenging energy embarrassment of data centres.
\\
\linebreak
The proposed adaptive optimisation approach for tasks scheduling is an advance of the energy-efficiency research. Although machine learning has been applied for efficiently optimising program in static environment\cite{wangf}, this will be the first step to apply machine learning algorithm for dynamic optimisation scheduler in multi-core based systems. Such an adaptive optimising approach will suit different platforms and fast promoting hardware design, i.e. the energy-efficiency problem can be solved for a certain long period. During the pursuing of the whole project, the work has to be conducted with data centres and companies. This will certainly enhance the bond between the companies and the university. The investigations around different task scheduling methods provide a guidance for further optimal scheduler problems in multi-core based systems.
