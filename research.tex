\section{Research Themes and Methodologies}
The primary goal of this PhD proposal is to investigate novel software techniques to improve the raw performance and energy efficiency of data centres. It will target data-intensive applications on heterogeneous many-core systems. It will demonstrate the performance portability for a set of applications across different hardware platforms (generations). These will be achieved by using automatic machine learning techniques to model the complex interactions between application workload and the underlying hardware, and to automatically derive optimisation heuristics for compilers and runtime systems.Concrete research results will include a new task and resource scheduler in Linux and a set of energy-aware compilation heuristics, targeting heterogeneous server computing environments. This work will be organised as three, interrelated themes. 

%The main goal of this PhD research is to develop the linux software system, which can realise automatically energy-efficiency data centres with improved performance, using machine learning algorithms. To achieve energy-efficiency of data centres, a smart adaptive scheduler is required to plan and distribute the optimal management of workload, physics resource and task-to-core mapping. The foundation of the proposed scheduler is to decide the essential resource with an accurate workload prediction. Based on an accurate workload estimation method, a smart tasks scheduler is needed to provide an optimal resource (energy and tasks) management of the data centre predictively. Due to the complexity of systems and heterogeneous hardware high-speed development, the key point here is using machine learning to ensure that \textbf{both workload estimator and tasks scheduler can adapt to changeable hardware and complex situations}. Moreover, the optimal scheduled programs have to be executed in an efficient way. Thus, the scheduled tasks need to be recompiled for fitting the optimal program-to-core mapping. Here a \textbf{dynamic recompiler using energy-aware heuristics and flexible abstractions} will be developed in the Jikes RVM~\cite{JRVM}.As a result, the proposed smart adaptive system can optimise the performance and energy-efficiency from hardware resource planning to software parallel execution automatically. Such portable optimising software, which is adaptive to new applications and promoting hardware, will produce reliable optimal performance for data centres. Considering such motivations, the work is planed into three themes.
\subsection{Theme 1: Server Workload Modelling and Prediction}
%Modelling\\
This theme aims to explore a predictive modelling based approach for resource scheduling. This will require the capability to accurately model and predict the server workload demands. For example, we will need to answer are we expecting heavy or light service requests in the next hour? An accurate workload prediction allows us to put servers that are unlikely to be utilised into sleep and wake them up ahead of time. This theme paves a stepping stone for theme 2. The output of this theme will be a set of models that can accurately predict the server workload demands for given application domains.
%
%With an accurate workload prediction, the data centre will cost just-sufficient power while confirming its computing performance. For various heterogeneous structures, a certain workload estimator has to be conducted with an efficient modelling function and an adaptive prediction algorithm.


%The optimal tasks scheduling is based on a precise estimation of system workload. With an accurate workload prediction, the data centre will cost just-sufficient power while confirming its computing performance. For various heterogeneous structures, a certain workload estimator has to be conducted with an efficient modelling function and an adaptive prediction algorithm.

\paragraph{Task 1.1: Server Workload Modelling} The first step is to discover the patterns of a given application domain. To do so, 
application traces will be collected for a certain period of time. Once there are sufficient data, statistically modelling techniques, 
such as clustering~\cite{} will be applied to discover the common, repeated patterns from the application traces. Because the trace 
collection is likely to be done in a production environment, trace collection must be low-overhead. This will require novel techniques to be constructed.

%Modelling task is the basement work of an workload estimator. Considering fast prediction, efficient models can help to save time for analysis and future computation. As introduced data centres are very complicated systems, it is valuable to develop an efficient model to express the workload, resource usage, and tasks runtime. Significantly, such a modelling method has to be available for different systems. In order to achieve an improving performance, the scheduling logs and power measurements need to be considered as key features in the model. Similarly, some researchers have developed task-scheduling approaches in multi-core processors through some system features, e.g. memory aware system~\cite{memaw} and energy aware system~\cite{game}.
%Prediction

\paragraph{Task 1.2: Server Workload Prediction} %In order to achieve high computing recourse utilisation, an accurate estimation of workload will be the foundation task of optimal tasks scheduling. The prediction accuracy needs to increase overtime with more system information collected. This is expected to achieve through machine learning algorithms, which can provide reliable probabilistic predictions with a certain error tolerance~\cite{ma}. For instance, I have applied decision trees to predict the system network usage using semi-supervised learning. The predition will be adaptive to different structure by the proposed wide-arrange applicable model.
Once the common, repeated workload patterns have been identified, we can then use statistical modelling techniques, such as regression models, to correlate those common patterns with certain characteristics of the application and users. The correlation we discover can then be used to build a model to predict the future workload demands by taking input the application characteristics and the previous workload information. The main challenge will be how to provide a certain degree of fault tolerance without being over-conservative. Given the uncertain nature of workload prediction, it would be interesting to see whether probabilistic methods will be useful for this problem~\cite{ma}.

\subsection{Theme 2: Predictive Modelling Based Resource Scheduling}
The proposed scheduler is expected to adapt to new changeable circumstances and various hardware structures. Motivated by high suitability of the proposed scheduler, I would like to employ machine learning algorithms to achieve predictively optimal scheduling for data centres. This theme will develop a plug-in machine learning based task scheduler in Linux。

\paragraph{Task 2.1: Adaptive optimal scheduler Implementation} In this task, the proposed tasks scheduling Linux software is required to manage the optimal physical recourse, plan the physics recourse, and provide program runtime scheduling, i.e. the functions include power modes of hardware, server time allocation, tasks distribution and migration. In order to achieve good performance with low power, the scheduler is required to handle a multi-objective optimisation. Thus, it is necessary to search around different appropriate system features as the criteria of the optimal scheduler design. The observability of system features (like energy~\cite{energy-aware}, execution time~\cite{multicore}, or memory usage~\cite{ma}) shows the possibilities to apply feedback methodology for optimal runtime scheduling. The appropriate system features can be the feedback cost function, which the optimisation programs learn to employ the minimising cost optimisation scheme using machine learning algorithm from. Similar methodology has been applied on smart electric grid management~\cite{powergrid} and distributed sensors optimisation problems~\cite{distributedsensors}. Thus, using reinforcement learning, the employed scheduler should adapt to new circumstances and continuously improve itself in the end-user's environment. The proposed feedback based optimal scheduler will be applied on real data centre systems for evaluation. Since the smart scheduler relays on workload estimation in the future, the predictive optimisation is expected to perform better then reactive schedulers, e.g.~\cite{energy-aware}.

\paragraph{Task 2.2 Investigations on scheduling methods for real applications} This tasks is to investigate how task scheduling schemes influence the management efficiency in terms of scheduling time consumption and tasks managing performance. The investigations here are trying to find a scheduling method, which can provide considerable optimisation result for energy-efficiency data centres in Linux. In order to evaluate different scheduling schemes, I will conduct around different machine learning algorithms (including supervised learning and reinforcement learning) and operation management methods (such as queuing theory, or game theory, e.g.~\cite{game}). To ensure the performance in reality, the scheduling methods have to be tested in real circumstances. The evaluation result will help the future research to choose appropriate management method on efficient tasks scheduling on not only data centres but also on other multi-core based systems.

\subsection{Theme 3: Energy-Aware Recompilation}
With the proposed intelligent scheduler, enormous simultaneously running tasks in data centres will be task-to-core mapped optimally. Motivated by a heuristic performance of data centres, the mapped programs should be recompiled and distributed into heterogeneous cores to achieve high execution performance within distributed recourse. The Dynamic recompiler should work for different programming languages and adapt to various scheduled recourse in terms of core computational ability, threads parallelism, and memory capacity. \FIXME{The output of this theme will be xxx}

\paragraph{Task 3.1: Flexible Language Abstractions}~The task will develop flexible language abstractions to support dynamic program reconfiguration for server applications. In Jikes RVM~\cite{JRVM} environment, the function will be built on existing languages and compiler frameworks, like OpenCL~\cite{opencl} and EPL~\cite{epl}. Extending from these languages and frameworks, the main task is to explore the trade-off between ease of programming and flexibilities of the abstractions. After investigating such trade-off, the milestone is to develop flexible language abstraction function, which can be exploited at runtime, with respect to programming ease and abstracting flexibility.

\paragraph{Task 3.2: Recompilation Strategies}~With the flexible language abstraction technique, the following task will conduct an energy-aware heuristics recompiling functions according to the proposed scheduling method. The key achievement is to realise an online recompiler which can utilise the performance of the heterogeneous hardware, like server cores or GPGPUs. Based on the flexible abstractions, the programs can be recompiled according to optimal scheduled recourse in the data centre. With the predictive optimal scheduler, the execution process is optimised at the same time. This scheduling-execution tool-chain technique will provide a smart adaptive approach to energy-efficient data centres with high performance.

