\section{Main Goal and Proposed Research}

The main goal of this PhD research is to develop the software, which can reduce the energy consumption while making programs faster, using machine learning algorithms. Considering such motivations, there are two main tasks to solve this problem: a) develop a precise workload prediction approach; b) intelligent tasks scheduler for optimising the computational resources usage. Due to the complexity of reality and high-speed hardware development, the key point here is ensuring those two functions to be adaptive to changeable hardware and complex situations. Moreover, the investigations around different modelling methods and scheduling approaches, which will help relative multi-core based system study, will be conducted as an important part. The detailed objectives are listed as following:
\begin{itemize}
\item
%Modelling 
Modelling the data centres systems. Efficient models help to save time for analysis and future computation. As introduced data centres are very complicated systems, it is valuable to develop an efficient model to express the workload, resource usage, and tasks runtime. Such a modelling method has to be available for different systems. In order to achieve an improving performance, the scheduling logs and power measurements need to be considered as key features in the model.  
\item
%Prediction
Workload Prediction approach development. In order to achieve high computing recourse utilisation, an accurate estimation of workload has to be developed as the foundation of optimal tasks scheduling. The predicting accuracy needs to increase overtime with more system information collected. This is expected to achieve through machine learning algorithms, which can provide reliable probabilistic predictions with a certain error tolerance.  
\item
%adaptive optimal tasks scheduler
Adaptive optimal tasks scheduler for data centres. The proposed  tasks scheduling software is required to manage the optimal physical recourse setting and provide program scheduling, including power modes of hardware, server time allocation, tasks distribution and migration. The employed machine learning method should adapt to new circumstances and continuously improve itself in the end-user's environment. The proposed adaptive optimal task scheduler can optimise the recourse use and improve the performance overtime, which will open up new possibilities and challenges.
\item
Investigations around different possible scheduling methods. With an appropriate model, task schedule schemes may influence the management efficiency in terms of scheduling time consumption and tasks managing performance. The investigations here are trying to find a scheduling method, which can provide considerable optimisation result for energy-efficiency data centres with high speed. The evaluation will be conducted around different machine learning algorithms (including supervised learning and reinforcement learning) and operation management methods (such as queuing theory, or game theory e.g.\cite{game}). The evaluation result will help the future research to choose appropriate management method on efficient tasks scheduling on not only data centres but also on other multi-core based systems. 
\end{itemize}
In order to achieve above tasks, there are several significant developments on scheduler design and brilliant methodologies recently. For instance, the state-of-the-art work, which models parallel tasks into a series of sequential or parallel segments, shows the possibilities of real-time scheduling in multi-core processor systems\cite{realtimeschedulea}\cite{realtimescheduleb}. Applying such scheduling method, huge programs are split into small tasks. In this way, the optimal runtime management is achievable by optimising each short computing time segment. The development on performance feature-awaring system realised energy-efficiency in specific multi-core processors, e.g. memory aware system\cite{memaw} and energy aware system\cite{energyaw}. As a consequence, the observability of system features show the possibilities to apply feedback methodology for optimal runtime scheduling. The energy consumption or memory can be the criteria data to train the machine learning algorithms. Such system features can be the cost function, which the optimisation programs learn to employ the minimising cost optimisation scheme using machine learning algorithm from. Similar methodology has been applied on smart electric grid management\cite{powergrid} and distributed sensors optimisation problems\cite{distributedsensors}. Thus, the feedback methodology is expected to implement optimal runtime scheduling with machine learning algorithms, like reinforcement learning.





