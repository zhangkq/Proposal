\section{Related Work}
\paragraph{~~~~Energy-efficiency issues for Data Centres}
%\FIXME{Describe the energy issues of data centres and existing hw/software
%techniques}
To reduce the huge power consumption, hardware engineers are striving to resolve the energy issue through power management techniques in processors~\cite{chanandopp}. They developed various power management features, like dynamic voltage, frequency scaling and heterogeneous processors, which have been applied in system components. Unfortunately, there are still no clear clues for software to realise the hardware potential~\cite{lookbackandfor,towardsenergyeff}. Therefore, almost 80\% of the data centres power usage is wasted~\cite{thecaseforepc}. To progress towards energy efficiency data centres, the proposed adaptive optimisation system for data centres that evolves and adapts to the changing workload and resource and delivers scalable energy-efficient performance is needed. Such an issue is expected to be resolved using an portable optimal approach employing the proposed adaptive scheduler (Theme 1 and Theme 2) and dynamic recompiler tools (Theme 3).


\paragraph{~~~~Heterogeneous Many-cores:} Heterogeneous many-cores, such as mixed CPU/GPU systems,
provide significant potential for
reducing energy consumption as certain devices can run some tasks more
efficiently~\cite{Cong:2012}. Currently, the task to core mapping process is
hard-corded when applications are built or is statically
determined when the program launches~\cite{cc11grewe}. Dynamically adjusting
the task to core mapping can improve energy efficiency~\cite{4550857}. However,
we must over come significant challenges to realise this benefit. Tasks must
be optimised for and scheduled onto different cores and core frequencies must
be set. Cores in a heterogeneous system have their own energy and performance
characteristics that will change on a per task basis. Schedules need to be
reconstructed as tasks enter, exit, or change phase. These challenges require
the OS to understand how the tasks will respond to different cores and the
compilation system to tailor the program dynamically in order to fit the
change of task to core mapping well. We will explore compiler and run-time
techniques that dynamically adjust the task to core mapping and resource
allocation and processor frequency in Theme 2.

\paragraph{~~~~Multi-tasking:} Multi-tasking brings a host of problems that we must
tackle. Tasks will have different resource requirements and will cause
interference with other tasks~\cite{zheng-cgo}. Prior work on compilers and
task mapping mainly targets single applications on unloaded machines without
external workloads~\cite{5375318,Raman:2012} or dedicate servers running one single
type of applications~\cite{Agarwal:2012:RDC:2228298.2228327}.
Energy and performance tradeoffs
on one task must not affect schedule fairness, throughput, responsiveness, etc.
for other tasks~\cite{lookbackandfor}. Moreover, tasks
will have different priorities so they will have different goals for the
energy/performance tradeoff. In Theme 2, we will build an energy efficient
multi-task scheduler. The scheduler will use the precise workload prediction
information from Theme 1 to predict the effect of different scheduling decisions
and to derive an optimised scheduling plan.

\paragraph{~~~~Evolving Computing Environment:} New architectures, application
domains and workloads constantly emerge, making it hard to keep optimisation
strategies current~\cite{5375327}. Any optimisation system must support fast,
easy re-tuning. Predictive modelling
has emerged as a viable means to automate the construction of optimisation
strategies for homogeneous multi-cores and single applications, outperforming
human experts, and across architectures~\cite{Stephenson:2003,Wang-pact}. The
predictive modelling techniques of Themes 1 and 2 will support models that can be
retrained whenever the environment changes.

\paragraph{~~~~Performance-Energy Tradeoff:} Saving energy without compromising
performance is increasingly difficult. Even in single task, single
core systems, energy and performance do not always correlate strongly~\cite{tradeoff}. 
One often cannot optimise code to maximise
performance \emph{and} to minimise energy
use~\cite{lookbackandfor}. Ideally, program optimisation would respond to changes in the environment, so that a server application, for example, would use different program versions depending on the program input, availabilities of hardware resources and other constraints like power cap and responsive time. The proposed Theme 2 should help to save energy while providing considerable performance. In addition, the problem will be pushed forward using flexible abstractions approach in Theme 3.
